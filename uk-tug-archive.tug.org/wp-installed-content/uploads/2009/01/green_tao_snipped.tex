\def\SNIP{\par\noindent[\ldots]\par} % added by J.Fine

\documentclass[12pt,a4paper,reqno]{amsart}
\usepackage{amsmath}
\usepackage{amsfonts}
\usepackage{amssymb}
%\usepackage{showkeys}  
% uncomment this when editing cross-references
\numberwithin{equation}{section}


     \addtolength{\textwidth}{3 truecm}
     \addtolength{\textheight}{1 truecm}
     \setlength{\voffset}{-.6 truecm}
     \setlength{\hoffset}{-1.3 truecm}
     
\theoremstyle{plain}

\newtheorem{theorem}[subsection]{Theorem}
\newtheorem{proposition}[subsection]{Proposition}
\newtheorem{lemma}[subsection]{Lemma}
\newtheorem{corollary}[subsection]{Corollary}
\newtheorem{claim}[subsection]{Claim}
\newtheorem{conjecture}[subsection]{Conjecture}
\newtheorem{question}[subsection]{Question}
\newtheorem{remark}[subsection]{Remark}

\theoremstyle{definition}

\newtheorem{definition}[subsection]{Definition}



%I prefer slanted leq and geq symbols
\renewcommand{\leq}{\leqslant}
\renewcommand{\geq}{\geqslant}

%This is all so I could finish a proof with an equation if need be.
\newsavebox{\proofbox}
\savebox{\proofbox}{\begin{picture}(7,7)%
  \put(0,0){\framebox(7,7){}}\end{picture}}
\def\boxeq{\tag*{\usebox{\proofbox}}}

%These functions are here because I find modular arithmetic difficult in LaTeX.
%Standard mod
\newcommand{\md}[1]{\ensuremath{(\mbox{mod}\, #1)}}
%For use in subscripts, e.g. in sums
\newcommand{\mdsub}[1]{\ensuremath{(\mbox{\scriptsize mod}\, #1)}}
%Standard mod for use in theorem environments
\newcommand{\mdlem}[1]{\ensuremath{(\mbox{\emph{mod}}\, #1)}}
%For use in subscripts in theorem environments
\newcommand{\mdsublem}[1]{\ensuremath{(\mbox{\scriptsize \emph{mod}}\, #1)}}

%Macros specifically for this paper
\def\vconst{\ensuremath \nu_{\mbox{\scriptsize const}}}
\def\vconstlem{\ensuremath \nu_{\mbox{\scriptsize \textup{const}}}}

\def\funif{\ensuremath f_{U}}
\def\fanti{\ensuremath f_{U^\perp}}

\def\intersect{\cap}
\def\Quad{\hbox{\rm Quad}}
\def\dim{\hbox{\rm dim}}
\def\cube{{\bf c}}
\def\Cube{{\mathcal C}}
\def\B{{\mathcal B}}
\def\D{{\mathcal D}}
\def\Z{{\mathbb Z}}
\def\E{{\mathbb E}}
\def\C{{\mathbb C}}
\def\R{{\mathbb R}}
\def\N{{\mathbb N}}
\def\Q{{\mathbb Q}}
\def\T{{\mathbb T}}
\def\x{{\bf x}}
\def\eps{{\varepsilon}}
\def\codim{\hbox{\rm codim}}
\def\w{w}
\def\Qmain{Q_{\mbox{\scriptsize main}}}
\def\rank{\hbox{\rm rank}}

\def\vs{\vspace{11pt}}
\def\ni{\noindent}

\def\proof{\noindent\textit{Proof. }}
\def\endproof{\hfill{\usebox{\proofbox}}}

\def\emph#1{{\it #1}}
\def\textbf#1{{\bf #1}}

     \begin{document}

\title{The primes contain arbitrarily long arithmetic progressions}

\author{Ben Green}
\address{School of Mathematics\\
University Walk\\
Bristol\\
BS8 1TW
}
\email{b.j.green@bristol.ac.uk}

\author{Terence Tao}
\address{Department of Mathematics\\University of California at Los Angeles\\ Los Angeles CA 90095}

\email{tao@math.ucla.edu}

\thanks{While this work was carried out the first author was a PIMS postdoctoral fellow at the University of British Columbia, Vancouver, Canada.  The second author was
a Clay Prize Fellow and was supported by a grant from the Packard Foundation.}

\subjclass{11N13, 11B25, 374A5}

\begin{abstract}
We prove that there are arbitrarily long arithmetic progressions of primes.\vs  

\ni There are three major ingredients. \SNIP

\end{abstract}


\maketitle

\SNIP \ni %
for all $x \in \Z_N$ (here $(m_0,t_0, L_0) = (3,2,1)$) and
\begin{eqnarray}\nonumber& & 
\E\bigg( \nu((x-y)/2) \nu((x-y+h_2)/2) \nu(-y) \nu(-y-h_1) \times \\ & & \nonumber
\qquad \times \; \nu((x-y')/2) \nu((x-y'+h_2)/2)  \nu(-y') \nu(-y'-h_1) \times \\ & &
\qquad \qquad \times \; \nu(x) \nu(x+h_1) \nu(x+h_2) \nu(x+h_1+h_2) \; 
\bigg| \; x,h_1,h_2,y,y' \in \Z_N\bigg) \nonumber \\ & & \qquad \qquad \qquad \qquad = 1 + o(1)\label{example-3}
\end{eqnarray} (here $(m_0, t_0, L_0) = (12,5,2)$).

\SNIP

\begin{proposition}[Generalised von Neumann]\label{vn} Suppose that $\nu$ is $k$-pseudorandom. Let $f_0, \ldots, f_{k-1} \in L^1(\Z_N)$ be functions which
are pointwise bounded by $\nu+\vconstlem$, or in other words
\begin{equation}\label{fj-bounds}
 |f_j(x)| \leq \nu(x) + 1 \hbox{ for all } x \in \mathbb{Z}_N, 0 \leq j \leq k-1.
\end{equation}
Let $c_0, \ldots, c_{k-1}$ be a permutation of $\{0,1,\ldots,k-1\}$ \textup{(}in practice we will take $c_j := j$\textup{)}. Then 
\[ \E\bigg( \prod_{j=0}^{k-1} f_j(x + c_j r) \;  \bigg| \; x,r \in \mathbb{Z}_N \bigg) = O\big(\inf_{0 \leq j \leq k-1} \| f_j \|_{U^{k-1}}\big) + o(1).\]
\end{proposition}

\SNIP


\providecommand{\bysame}{\leavevmode\hbox to3em{\hrulefill}\thinspace}
\providecommand{\MR}{\relax\ifhmode\unskip\space\fi MR }
% \MRhref is called by the amsart/book/proc definition of \MR.
\providecommand{\MRhref}[2]{%
  \href{http://www.ams.org/mathscinet-getitem?mr=#1}{#2}
}
\providecommand{\href}[2]{#2}
\begin{thebibliography}{10}

\bibitem{assani}
I. Assani, \emph{Pointwise convergence of ergodic averages along cubes}, preprint.

\bibitem{balog1} A. Balog, \emph{Linear equations in primes,} Mathematika \textbf{39} (1992) 367--378.

\bibitem{balog2} \bysame, \emph{Six primes and an almost prime in four linear equations,} Can. J. Math. \textbf{50} (1998), 465--486.

\bibitem{bergelson-leibman} V. Bergelson and A. Leibman, \emph{Polynomial extensions of van der Waerden's and Szemer\'edi's theorems,} J. Amer. Math. Soc. \textbf{9} (1996), 725--753.

\SNIP

\end{thebibliography}

\end{document}

